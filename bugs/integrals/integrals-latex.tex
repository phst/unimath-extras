\documentclass{article}

\usepackage{xparse}

\ExplSyntaxOn

\luatex_if_engine:TF {
  \NewDocumentCommand \ConTeXtInt { } { \luatexUmathchar "1 "0 "222B \nolimits }
  \use_i:nn
} {
  \NewDocumentCommand \ConTeXtInt { } { }
  \xetex_if_engine:TF {
    \use_i:nn
  } {
    \use_ii:nn
  }
} {
  \usepackage { unicode-math }
  \setmathfont { Cambria~ Math }
} {
  \usepackage { amsmath }
}

\luatex_if_engine:T {
  \luatexUmathoperatorsize\displaystyle=13pt
}

\ExplSyntaxOff


\begin{document}

\texttt{\string\intop\ =\ \meaning\intop\\\string\int\ =\ \meaning\int}

\[ \displaystyle \intop \int \intop\nolimits \intop\limits \intop\displaylimits \ConTeXtInt \iint \iiint \iiiint \]

\fontname\textfont4

\end{document}
